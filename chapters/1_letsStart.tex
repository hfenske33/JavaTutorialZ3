\definecolor{dkgreen}{rgb}{0,0.6,0}
\definecolor{gray}{rgb}{0.5,0.5,0.5}
\definecolor{mauve}{rgb}{0.58,0,0.82}

\lstset{frame=tb,
  language=Java,
  aboveskip=3mm,
  belowskip=3mm,
  stepnumber=1,
  showstringspaces=false,
  columns=flexible,
  basicstyle={\small\ttfamily},
  %numbers=left,
  numberstyle=\tiny\color{gray},
  keywordstyle=\color{blue},
  commentstyle=\color{dkgreen},
  stringstyle=\color{mauve},
  breaklines=true,
  breakatwhitespace=true,
  tabsize=3
}



\chapter{Getting Started}

For the following examples here is the main-structure of the use of the solver: Create a context...Do something...Close the context.\\
\begin{lstlisting}[frame=single]
HashMap<String, String> cfg = new HashMap<String, String>();
cfg.put("proof", "true");
Context ctx = new Context(cfg);
//some code
ctx.close();
\end{lstlisting}


\section{Example 1 - Add two integers}
Let's start with a simple formula. We just want to express x + y with z3.
\begin{lstlisting}[frame=single]
HashMap<String, String> cfg = new HashMap<String, String>();
cfg.put("proof", "true");
Context ctx = new Context(cfg);

IntExpr x = ctx.mkIntConst("x");
IntExpr y = ctx.mkIntConst("y");
Expr e = ctx.mkAdd(x, y);
System.out.println(e);

ctx.close();
\end{lstlisting}

The output is (+ x y). 
That's not very exciting. 
The output is just what we already knew. Let's go one step further and add two numbers.
Now we define two numbers (three, five) and add them together.

\begin{lstlisting}[frame=single]
HashMap<String, String> cfg = new HashMap<String, String>();
cfg.put("proof", "true");
Context ctx = new Context(cfg);

IntExpr three = ctx.mkInt(3);
IntExpr five = ctx.mkInt(5);
Expr e = ctx.mkAdd(three, five);
Solver s = ctx.mkSolver();
		
if(s.check() == Status.SATISFIABLE) {
	System.out.println(s.getModel().evaluate(e, true));
}else {
	System.out.println(s + " is not solvable");
}

ctx.close();
\end{lstlisting}

The output is of course 8.
s.check() is not really necessary in this case, but you should not miss it if you want to do an evaluation. 
s.check() only asks if the calculus is solvable. 
This question makes more sense in the next example...


\section{Example 2 - Add two integers with assertions}

Let's assume that we want to add two numbers x and y, with a destricted domain:
\begin{enumerate}
\item x < 3	
\item y > 5	
\item x + 2*y = 14 
\end{enumerate}

\begin{lstlisting}[frame=single]
HashMap<String, String> cfg = new HashMap<String, String>();
cfg.put("proof", "true");
Context ctx = new Context(cfg);

IntExpr x = ctx.mkIntConst("x");
IntExpr y = ctx.mkIntConst("y");
IntExpr three = ctx.mkInt(3);
IntExpr five = ctx.mkInt(5);
Expr e = ctx.mkAdd(x, y);
		
BoolExpr b1 = ctx.mkLt(x, three);
BoolExpr b2 = ctx.mkGt(y, five);
BoolExpr b3 = ctx.mkEq(ctx.mkAdd(x, ctx.mkMul(ctx.mkInt(2), y)), ctx.mkInt(14));
BoolExpr bx = ctx.mkAnd(b1, b2, b3);
		
Solver s = ctx.mkSolver();
s.add(bx);
	
if(s.check() == Status.SATISFIABLE) {
	System.out.println(s.getModel().evaluate(e, true));
}else {
	System.out.println(s + " is not solvable");
}

ctx.close();
\end{lstlisting}
Of course you do not have to summarize the assertions in one assertion \textit{bx}. One could also write: \textit{s.add (b1); s.add (b2); s.add (b3);}\\
Or you can put everything together and write:\\\\
\textit{	s.add(ctx.mkAnd(ctx.mkLt(x, three), ctx.mkGt(y, five), ctx.mkEq(ctx.mkAdd\\
(x, ctx.mkMul(ctx.mkInt(2), y)), ctx.mkInt(14))));}.\\\\
One could add a \textit{System.out.println(s)}.
The output looks like this:\\
(declare-fun y () Int)\\
(declare-fun x () Int)\\
(assert (and (< x 3) (> y 5) (= (+ x (* 2 y)) 14)))\\


