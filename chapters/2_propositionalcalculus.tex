\definecolor{dkgreen}{rgb}{0,0.6,0}
\definecolor{gray}{rgb}{0.5,0.5,0.5}
\definecolor{mauve}{rgb}{0.58,0,0.82}

\lstset{frame=tb,
  language=Java,
  aboveskip=3mm,
  belowskip=3mm,
  stepnumber=1,
  showstringspaces=false,
  columns=flexible,
  basicstyle={\small\ttfamily},
  %numbers=left,
  numberstyle=\tiny\color{gray},
  keywordstyle=\color{blue},
  commentstyle=\color{dkgreen},
  stringstyle=\color{mauve},
  breaklines=true,
  breakatwhitespace=true,
  tabsize=3
}



\chapter{Propositional Calculus}


\section{Example 3 -Transitive Relation}
\begin{lstlisting}[frame=single]
HashMap<String, String> cfg = new HashMap<String, String>();
cfg.put("proof", "true");
Context ctx = new Context(cfg);

IntExpr x = (IntExpr) ctx.mkConst("x", ctx.getIntSort());
IntExpr y = (IntExpr) ctx.mkConst("y", ctx.getIntSort());
IntExpr z = (IntExpr) ctx.mkConst("z", ctx.getIntSort());
		
BoolExpr b1 = (ctx.mkEq(x, y));
BoolExpr b2 = (ctx.mkEq(y, z));
BoolExpr b3 = (ctx.mkEq(x, y));
BoolExpr bx1 = ctx.mkImplies((ctx.mkAnd(b1, b2)), b3);
BoolExpr bx2 = ctx.mkImplies((ctx.mkAnd(b1, b2)), ctx.mkNot(b3));
		
Solver s = ctx.mkSolver();
if (s.check() == Status.SATISFIABLE) {
	System.out.print(bx1);
	System.out.println(s.getModel().evaluate(bx1, true));
	System.out.print(bx2);
	System.out.println(s.getModel().evaluate(bx2, true));
} else {
	System.out.println(s + " is not solvable");
}

ctx.close();
\end{lstlisting}


\section{Example 4 - Quantifiers I}


\section{Example 5 - Quantifiers II - UNSATISFIABLE?}
\begin{lstlisting}[frame=single]
HashMap<String, String> cfg = new HashMap<String, String>();
cfg.put("proof", "true");
Context ctx = new Context(cfg);

Sort[] sorts = new Sort[2];
sorts[0] = (ctx.getIntSort());
sorts[1] = (ctx.getIntSort());

Symbol[] names = new Symbol[2];
names[0] = ctx.mkSymbol("x");
names[1] = ctx.mkSymbol("y");
		
IntExpr[] vars = new IntExpr[2];
vars[0] = (IntExpr) ctx.mkBound(0, sorts[0]);
		
IntExpr x = (IntExpr) ctx.mkConst("x", ctx.getIntSort());
IntExpr y = (IntExpr) ctx.mkConst("y", ctx.getIntSort());
		
// x + y >= 12
BoolExpr b1 = ctx.mkGe(ctx.mkAdd(x, y), ctx.mkInt(12));
// x <= 6
BoolExpr b2 = ctx.mkLe(x, ctx.mkInt(6));
// y < 7
BoolExpr b3 = ctx.mkLt(y, ctx.mkInt(4));
// y != x
BoolExpr b4 = ctx.mkDistinct(x, y);
// b1 && b2 && b3 && b4
BoolExpr bx = ctx.mkAnd(b1, b2, b3, b4);		
BoolExpr exForAll = ctx.mkForall(sorts, names, bx, 1, null, null, null, null);
		
System.out.println("Quantifier X Y: " + exForAll.toString());
Solver s = ctx.mkSolver();
s.add(exForAll);
Status status = s.check();
System.out.println(status);

ctx.close();
\end{lstlisting}

\section{Example 6 - Quantifiers III}